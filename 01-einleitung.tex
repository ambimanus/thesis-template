%%%%%%%%%%%%%%%%%%%%%%%%%%%%%%%%%%%%%%%%%%%%%%%%%%%%%%%%%%%%%%%%%%%%%%%%%%%%%%%
% Autor: Christian Hinrichs, Januar 2014
%%%%%%%%%%%%%%%%%%%%%%%%%%%%%%%%%%%%%%%%%%%%%%%%%%%%%%%%%%%%%%%%%%%%%%%%%%%%%%%
% Setup
%
% Der folgende Code wird nur ausgeführt, wenn diese Datei direkt kompiliert
% wird (also ohne main.tex):
\ifx\fulldocument\undefined
  %%%%%%%%%%%%%%%%%%%%%%%%%%%%%%%%%%%%%%%%%%%%%%%%%%%%%%%%%%%%%%%%%%%%%%%%%%%%%%%
% Autor: Christian Hinrichs, 2013--2015
%%%%%%%%%%%%%%%%%%%%%%%%%%%%%%%%%%%%%%%%%%%%%%%%%%%%%%%%%%%%%%%%%%%%%%%%%%%%%%%
\documentclass[
  BCOR=10mm,                                                                     % ca. 280 Seiten = 140 Blätter * 0.14 mm / 2 = Bindekorrektur 10 mm
  DIV=9,                                                                        % Textspiegel: klassische Neunerteilung
  twoside=true,                                                                 % Dokument zweiseitig setzen?
  headinclude=true,                                                             % Kopfzeile zum Textkörper zählen?
  footinclude=false,                                                            % Fußzeile zum Textkörper zählen?
  paper=A4,                                                                     % Papierformat zur Berechnung des Satzspiegels
  pagesize=auto,                                                                % Papierformat des Kompilates gemäß 'paper' setzen
  % draft=true,                                                                 % Entwurfsmodus, z.B. mit hervorgehobenen overfull-boxes
  parskip=false,                                                                % Absatzauszeichnung: 1 em Einzug, kein Abstand
  cleardoublepage=empty,                                                        % Vakatseiten komplett leer
  open=right,                                                                   % Kapitel immer auf rechter Doppelseite beginnen
  chapterprefix=false,                                                          % Präfix "Kapitel 1" vor Kapitelüberschrift?
  appendixprefix=false,                                                         % Präfix "Anhang A" vor Anhangüberschrift?
  numbers=noenddot,                                                             % Gliederungsnummern mit Punkt? (auto = Duden-Vorschrift)
  % leqno,                                                                        % Gleichungen links statt rechts nummerieren
  % fleqn,                                                                        % Gleichungen linksbündig statt zentriert ausgeben
  toc=bib,                                                                      % Literaturverzeichnis im Inhaltsverzeichnis
  toc=listof,                                                                   % Abbildungs- und Tabellenverzeichnis (so vorhanden) im Inhaltsverzeichnis
]{scrreprt}                                                                     % scrreprt, da scrbook keine abstract-Umgebung hat
%
% -----------------------------------------------------------------------------
% Kodierung
%
\usepackage[utf8]{inputenc}                                                     % Aktiviere UTF-8
%
% -----------------------------------------------------------------------------
% Sprache
%
\usepackage[ngerman,english]{babel}
\selectlanguage{ngerman}
%
% -----------------------------------------------------------------------------
% Entwurfsmodus
%
\KOMAoption{draft}{false}                                                       % Entwurfsmodus?
\def\printversion{false}                                                        % Printversion (alle Hyperlinks schwarz-weiß)?
\def\versionlabel{true}                                                         % Versionsangabe im footer?
%
%
\usepackage{scrtime}
\usepackage{ifthen}
\newboolean{printversion}
\setboolean{printversion}{\printversion}
%
\newcommand{\finalVersionString}{}
\ifthenelse{\boolean{\versionlabel}}{%
  \usepackage[draft]{prelim2e}
      \renewcommand{\PrelimWords}{\relax}
      \renewcommand{\PrelimText}{%
        \color{halfgray}\tiny[\,Version: \today, \thistime\ Uhr\,]%
      }
      \renewcommand{\finalVersionString}{Entwurf, \today}
}{
  \renewcommand{\finalVersionString}{Eingereichte Version, \today}
}
%
% -----------------------------------------------------------------------------
% Fonts
%
% Nutze "Type 1" Font anstelle des alten OT1
\usepackage[T1]{fontenc}
%
% Lade Charter für Fließtext + Mathe-Modus
% (rmfamily wird hierdurch auf mdbch gesetzt)
\usepackage[charter]{mathdesign}
%
% Lade Kp-Fonts mit allen optionen deaktiviert, damit der Font verfügbar ist,
% aber momentant noch nicht angewendet wird. Dies soll selektiv geschehen,
% wie bspw. die Verwendung der Mengensymbole (\mathsfit, siehe style.tex).
\usepackage[nomath,notext,noamsmath,notextcomp]{kpfonts}
% Setze Kp-Fonts als Standard für sffamily (z.B. Überschriften)
\renewcommand{\sfdefault}{jkpss}
%
% TODO: Monospace Font? (Z.B. Source Sans)
%
% Eigentlich sind alle Mathesymbole etc. durch mathdesign bereits verfügbar,
% aber amsmath wird dennoch benötigt, bspw. für die {equation*} Umgebung.
\usepackage{amsmath}
% Note that the amsmath package sets \interdisplaylinepenalty to 10000
% thus preventing page breaks from occurring within multiline equations. Use:
% \interdisplaylinepenalty=2500
% after loading amsmath to restore such page breaks.
%
% Lade fehlende Symbole, wie bspw. textbullet in "sl" shape und korrekter Größe
\usepackage{textcomp}
%
%
% Größe der Grundschrift
\KOMAoption{fontsize}{11pt}
%
% Durchschuss erhöhen
\usepackage{setspace}
\setstretch{1.15}
%
% TODO: Der Durchschuss sollte für die Titelseite wieder auf den Normalwert
% gesetzt werden, siehe scrguide S. 41 f.
%
%
% Abschließend Satzspiegel neu berechnen!
\recalctypearea
%
% -----------------------------------------------------------------------------
% Microtype
% (siehe www.khirevich.com/latex/microtype)
%
\usepackage[
  activate={true,nocompatibility},                                              % activate protrusion and expansion
  final,                                                                        % enable microtype; use "draft" to disable
  tracking=true,                                                                % activate tracking
  kerning=true,                                                                 % activate kerning
  spacing=true,                                                                 % activate spacing
  factor=1100,                                                                  % add 10% to the protrusion amount (default is 1000)
  stretch=10,                                                                   % reduce stretchability (default is 20)
  shrink=10,                                                                    % reduce shrinkability (default is 20)
]{microtype}
%
\SetExtraKerning[unit=space]
  {
    encoding={*},
    family={bch},
    series={*},
    size={footnotesize,small,normalsize}
  }
  {
    \textendash={400,400},                                                      % en-dash, add more space around it
       "28={ ,150},                                                             % left bracket, add space from right
       "29={150, },                                                             % right bracket, add space from left
       \textquotedblleft={ ,150},                                               % left quotation mark, space from right
       \textquotedblright={150, }                                               % right quotation mark, space from left
  }
%
% -----------------------------------------------------------------------------
% classicthesis & arsclassica
%
\usepackage{textcase}
%
% ************************************************************
% Colors (both)
% ************************************************************
\usepackage{xcolor}
\definecolor{halfgray}{gray}{0.55}
\definecolor{HALFGRAY}{gray}{0.55}
\definecolor{webbrown}{rgb}{.6,0,0}
\definecolor{RoyalBlue}{cmyk}{1, 0.50, 0, 0}
%
% http://paletton.com/#uid=73C0u0kqdnZgRvslGrjthjD-Yeu
\definecolor{primary}{HTML}{1F4A7D}
\definecolor{primaryA}{HTML}{94A4B6}
\definecolor{primaryB}{HTML}{6581A4}
\definecolor{primaryC}{HTML}{29415E}
\definecolor{primaryD}{HTML}{0A2A51}
\definecolor{secondaryA}{HTML}{BF9D23}
\definecolor{secondaryB}{HTML}{BF9D23}
\definecolor{complementary}{HTML}{BF5B23}
%
\definecolor{darkgray}{gray}{0.25}
\definecolor{lightgray}{gray}{0.66}
%
% ************************************************************
% Fancy stuff (arsclassica)
% ************************************************************
\SetTracking[context=trackinglarge]{encoding = *}{160}                          % Definiere Kontext: 160er tracking
\SetTracking[context=trackingsmall]{encoding = *}{80}                           % Definiere Kontext: 80er tracking
%
\DeclareRobustCommand{\spacedallcaps}[1]{%                                      % Capitals (uppercase) mit 160er spacing
  \microtypesetup{expansion=false}%
  \fontfamily{jkpss}\lsstyle\microtypecontext{tracking=trackingsmall}%
  \MakeTextUppercase{#1}%
}
\DeclareRobustCommand{\spacedlowsmallcaps}[1]{%                                 % Capitals (lowercase) mit 80er spacing
  \microtypesetup{expansion=false}%
  \fontfamily{jkpss}\lsstyle\microtypecontext{tracking=trackingsmall}%
  \fontshape{sc}\selectfont\MakeTextLowercase{#1}%
}
%
% ************************************************************
% Headlines (arsclassica)
% ************************************************************
\usepackage[automark]{scrpage2}
\clearscrheadings
%
\renewcommand{\chaptermark}[1]{\markleft{%                                      % Kapitelname ohne Nummer, in small caps
  {\spacedlowsmallcaps{#1}}%
}}
%
\renewcommand{\sectionmark}[1]{\markright{%                                     % Sectionname mit Nummer, in small caps
  {{\small\thesection} \spacedlowsmallcaps{#1}}%
  % TODO: sectionmark bei neuem kapitel löschen?
}}
%
\lehead{\mbox{%                                                                 % Header oben links: Seitenzahl, vert. Linie, Kapitelname
  \llap{\small\pagemark\kern1em\color{halfgray}\vline}%
  \color{halfgray}\hspace{0.5em}\headmark\hfil%
}}
%
\rohead{\mbox{%                                                                 % Header oben rechts: Section, vert. Linie, Seitenzahl
  \hfil{\color{halfgray}\headmark\hspace{0.5em}}%
  \rlap{\small{\color{halfgray}\vline}\kern1em\pagemark}%
}}
%
\ofoot[\relax]{%                                                                % Keine Seitenzahlen in Fußzeile,
  \relax%                                                                       % auch nicht bei Kapitelanfang
}
%
\setkomafont{pageheadfoot}{%                                                    % Schriftart für Header-Text
  \normalfont\fontfamily{jkpss}\fontshape{sc}\selectfont%
}
%
\setkomafont{pagenumber}{%                                                      % Schriftart für Seitenzahlen
  \normalfont\small\fontfamily{jkpss}\selectfont%
}
%
%************************************************************
% Layout of the chapter-, section-, subsection-,
% subsubsection-, paragraph and description-headings (arsclassica)
%************************************************************
\newfont{\chapterNumber}{eurb10 scaled 5000}                                    % Lade skalierte Euler-Font für Kapitelnummer
%
\renewcommand*{\chapterformat}{%                                                % Große graue Euler-Kapitelnummer mit vert. Linie
  {%
  \color{halfgray}%
  \chapterNumber\thechapter%
  }%
  \hspace{15pt}\smash{\protect\rule[-7pt]{0.5pt}{42pt}}\hspace{15pt}%
}
%
\renewcommand*{\othersectionlevelsformat}[3]{%                                  % Stil der Abschnittsnummerierung
  \fontfamily{jkpss}\fontshape{sc}\selectfont%
  \MakeTextLowercase{%
    #3\autodot\enskip
  }%
}
%
\setkomafont{chapter}{%                                                         % Stil der Kapitelüberschriften
  \microtypecontext{tracking=trackinglarge}%
  \normalfont\Large\fontfamily{jkpss}%
  \lsstyle%
  \MakeTextUppercase%
}
%
\setkomafont{section}{%                                                         % Stil der Sectionüberschriften
  \microtypecontext{tracking=trackingsmall}%
  \normalfont\Large\fontfamily{jkpss}\fontshape{sc}\selectfont%
  \lsstyle%
  % \lowercase%           <-- This would break UTF-8 chars...
}
%
\setkomafont{subsection}{%                                                      % Stil der Subsectionüberschriften
  \normalfont\normalsize\fontfamily{jkpss}\selectfont%
}
%
\setkomafont{subsubsection}{%                                                   % Stil der Subsubsectionüberschriften
  \normalfont\normalsize\itshape\fontfamily{jkpss}\selectfont%
}
%
\setkomafont{paragraph}{%                                                       % Stil der Paragraphüberschriften
  \normalfont\normalsize\bfseries\fontfamily{jkpss}\selectfont%
}
%
% ***********************************************************
% Misc
% ***********************************************************
%
% % Disable single lines at the start of a paragraph (Schusterjungen)
% \clubpenalty = 10000
% % Disable single lines at the end of a paragraph (Hurenkinder)
% \widowpenalty = 10000
% \displaywidowpenalty = 10000 % formulas
%
% % Enumeration environment with small caps
% \newenvironment{aenumerate}
%     {\def\theenumi{\textsc{\alph{enumi}}}%
%      \enumerate}
%     {\endenumerate}
%
% hack to get the content headlines right
% \def\toc@headingbkORrp{%
%   \def\toc@heading{%
%     \chapter*{\contentsname}%
%     \@mkboth{\spacedlowsmallcaps{\contentsname}}
%       {\spacedlowsmallcaps{\contentsname}}}}
% \@ifclassloaded{scrreprt}{\toc@headingbkORrp}{}
% \@ifclassloaded{scrbook}{\toc@headingbkORrp}{}
%
%************************************************************
% lists
%************************************************************
% \renewcommand\labelitemi{\color{halfgray}$\bullet$}                             % Semi-transparent bullet points
% \let\oldlabelenumi\labelenumi
% \renewcommand\labelenumi{\color{halfgray}\oldlabelenumi}                        % Semi-transparent numbers
%
%************************************************************
% caption
%************************************************************
\addtokomafont{captionlabel}{\bfseries}                                         % Caption labels in bold face
%
% ***********************************************************
% layout of the TOC, LOF and LOT (LOL-workaround see next section)
% ***********************************************************
% TODO (siehe classicthesis.sty)
%
% -----------------------------------------------------------------------------
% Abbildungen
%
\setcapindent{1em}                                                              % Reduzierter hängender Einzug (siehe scrguide)
% -----------------------------------------------------------------------------
% Tabellen
%
\usepackage{array}                                                              % liefert \newcolumntype
\usepackage{booktabs}                                                           % Typografisch korrekte Tabellen
%
% Guidelines:
%
% 1. Never, ever use vertical rules.
% 2. Never use double rules.
% 3. Put the units in the column heading (not in the body of the table).
% 4. Always precede a decimal point by a digit; thus 0.1 not just .1.
% 5. Do not use ‘ditto’ signs or any other such convention to repeat a previous
%    value. In many circumstances a blank will serve just as well. If it won’t,
%    then repeat the value.
%
% -----------------------------------------------------------------------------
% Algorithmen + Code
%
% Das Paket 'algorithm' liefert die 'algorithm' Umgebung (floating)
\usepackage{algorithm}
\floatname{algorithm}{Algorithmus}
\newcommand{\AND}{\textbf{and}}
\newcommand{\OR}{\textbf{or}}
%
% Das Paket 'algpseudocode' wird benutzt, um *Algorithmen* zu formatieren.
\usepackage{algpseudocode}
%
% Das Paket 'listings' wird benutzt, um *Quellcode* zu formatieren.
\usepackage{listings}
%
% -----------------------------------------------------------------------------
% Anführungszeichen
%
\usepackage[
  strict,                                                                       % Warnungen werden als Fehler ausgegeben
  babel                                                                         % Unterstützung für babel aktivieren
]{csquotes}
%
% -----------------------------------------------------------------------------
% Literaturverzeichnis
%
\usepackage[
  style=alphabetic,                                                             % Stil (cites + bib)
  % sorting=nty,                                                                  % Sortierung: Name, Titel, Jahr
  sortlocale=de,                                                                % Deutsche Sprache
  sortcase=false,                                                               % Sortierung abhängig von Groß-/Kleinschreibung
  sortcites=false,                                                              % Sortiere multiple Zitate gemäß Literaturverzeichnis
  % maxnames=3,                                                                   % Max. Anzahl Namen bevor Abkürzung (cites + bib)
  % minnames=1,                                                                   % Min. Anzahl Namen bevor Abkürzung (cites + bib)
  maxbibnames=3,                                                                % Max. Anzahl Namen bevor Abkürzung (bib)
  % minbibnames=1,                                                                % Min. Anzahl Namen bevor Abkürzung (bib)
  % maxcitenames=2,                                                               % Max. Anzahl Namen bevor Abkürzung (cites)
  % mincitenames=1,                                                               % Min. Anzahl Namen bevor Abkürzung (cites)
  % maxitems=3,                                                                   % Max. Anzahl Items in Listen bevor Abkürzung (cites + bib)
  % minitems=1,                                                                   % Min. Anzahl Items in Listen bevor Abkürzung (cites + bib)
  % autocite=inline,                                                              % Verhalten von \autocite
  % autopunct=true,                                                               % Beachte mögliche Punktuation nach Zitat?
  % block=none,                                                                   % Abstand zwischen bib-Einträgen
  % hyperref=true,                                                                % Unterstützung für hyperref?
  backref=true,                                                                   % Aktiviere Back-References?
  % backrefstyle=three,                                                           % Fasse aufeinanderfolgende Seitenzahlen zusammen
  % indexing=false,                                                               % Aktiviere Index-Unterstützung für Cites/Bibs?
  % abbreviate=true,                                                              % Aktiviere Abkürzungen?
  % date=comp,                                                                    % Abkürzungsstil für Datum
  backend=biber,                                                                  % Nutze das mächtige biber als backend
  % texencoding=auto,                                                             % Encoding der tex Dateien
  % bibencoding=auto,                                                             % Encoding der bib Dateien
  isbn=false,                                                                     % Zeige ISBN?
  url=false,                                                                      % Zeige URL?
  % doi=true,                                                                     % Zeige DOI?
  % eprint=true,                                                                  % Zeige eprint?
  useprefix=false,                                                                % Namens-Präfix (z.B. "von") Teil des Nachnamens?
]{biblatex}
%
% \setlength{\bibitemsep}{1em}                                                    % Abstände zwischen Literatureinträgen
% \setlength{\bibhang}{0em}                                                       % Überhängende Einrückung
\urlstyle{same}                                                                 % Schriftart der DOIs soll sich nicht unterscheiden
\renewcommand{\finentrypunct}{}                                                 % removes period at the very end of bibliographic record
%
\addbibresource{literatur.bib}                                                  % Lade Literatur-DB
% \addbibresource{misc.bib}                                                       % Lade weitere Literatur-DB
%
% Zusätzlicher Eintragstyp @standard, Beispiel:
% @standard{VDI4655,
%   organization = {{Verein Deutscher Ingenieure}},
%   title = {Referenzlastprofile von Ein- und Mehrfamilienhäusern für den Einsatz von KWK-Anlagen},
%   type = {VDI Richtlinie},
%   number = {VDI 4655},
%   year = {2008},
%   shorthand = {VDI4655},
% }
\DeclareBibliographyDriver{standard}{%
  \usebibmacro{bibindex}%
  \usebibmacro{begentry}%
  \usebibmacro{author}%
  \setunit{\labelnamepunct}\newblock
  \usebibmacro{title}%
  \newunit\newblock
  \printfield{number}%
  \setunit{\addspace}\newblock
  \printfield[parens]{type}%
  \newunit\newblock
  \usebibmacro{organization+location+date}%
  \newunit\newblock
  \iftoggle{bbx:url}
    {\usebibmacro{url+urldate}}
    {}%
  \newunit\newblock
  \usebibmacro{addendum+pubstate}%
  \setunit{\bibpagerefpunct}\newblock
  \usebibmacro{pageref}%
  % \newunit\newblock
  % \usebibmacro{related}%
  \usebibmacro{finentry}%
}
%
% -----------------------------------------------------------------------------
% Weitere Pakete
%
\usepackage{pdfpages}                                                           % Für Titelseite aus separatem Dokument
%
\usepackage{graphicx}
\graphicspath{{images/}}                                                        % Setze Bild-Pfad
\usepackage[tight]{subfigure}
% \let\thesubfigureorig\thesubfigure                                              % Speichere aktuelles Format der Nummerierung
\usepackage{amsthm}                                                             % verbessert \newtheorem, für thmtools
\usepackage{mdframed}                                                           % Rahmen, für thmtools
\usepackage{thmtools}                                                           % liefert \declaretheorem
\usepackage{environ}
%
\usepackage{tikz}                                                               % Diagramme
\usetikzlibrary{
  trees,
  positioning,
  patterns,
  calc,
  intersections,
  shapes,
  arrows,
  decorations.markings,
  decorations.pathmorphing,
  decorations.pathreplacing,
}
%
%
% Key-Value system
% http://tex.stackexchange.com/a/37113
\newcommand{\setvalue}[1]{\pgfkeys{/variables, #1}}
\newcommand{\getvalue}[1]{\pgfkeysvalueof{/variables/#1}}
\newcommand{\declare}[1]{%
 \pgfkeys{
  /variables/#1.is family,
  /variables/#1.unknown/.style = {\pgfkeyscurrentpath/\pgfkeyscurrentname/.initial = ##1}
 }%
}
\declare{}
% Examples:
% \begin{document}
%   \setvalue{VARIABLE1 = foo foo bar}
%   \getvalue{VARIABLE1}
%   \declare{test/}
%   \setvalue{test/property = 12}
%   \getvalue{test/property}
% \end{document}
%
\usetikzlibrary{external}                                                       % Externalize TikZ images
\tikzexternalize[prefix=images/,mode=list and make,optimize=false]
%
%
\usepackage[german,intoc,noprefix]{nomencl}
\makenomenclature
% Mehrere nomencl-Abschnitte:
% http://www.mrunix.de/forums/showpost.php?p=210422&postcount=29
\renewcommand{\nomname}{Akronyme}
\renewcommand{\nompreamble}{\markboth{\nomname}{\nomname}}
\renewcommand{\nomlabelwidth}{2.5cm}
\newcommand{\nomaltname}{Symbole}
\newcommand{\nomaltpreamble}{\markboth{\nomaltname}{\nomaltname}}
\newcommand{\nomaltpostamble}{}
\newcommand{\switchnomitem}{S}
\renewcommand{\nomgroup}[1]{%
  \ifthenelse{\equal{#1}{\switchnomitem}}{\switchnomalt}{}}
\newcommand{\switchnomalt}{%
  \end{thenomenclature}
  \renewcommand{\nomname}{\nomaltname}
  \renewcommand{\nompreamble}{\nomaltpreamble}
  \renewcommand{\nompostamble}{\nomaltpostamble}
  \begin{thenomenclature}
}
%
%
% Unit measurement system:
% \usepackage{fp}
% \newlength{\TOarg} \newlength{\TOunit}
% {\catcode`p=12 \catcode`t=12 \gdef\TOnum#1pt{#1}}
% \newcommand\TOop[2]{\setlength{\TOarg}{#2}%
%    \FPdiv\TOres{\expandafter\TOnum\the\TOarg}{\expandafter\TOnum\the\TOunit}%
%    \FPround\TOres\TOres{#1}}
% \newcommand{\TOspace}{\ }
% \newcommand\TOpt[2][2]{\setlength{\TOunit}{1pt}\TOop{#1}{#2}\TOres\TOspace pt}
% \newcommand\TOin[2][2]{\setlength{\TOunit}{1in}\TOop{#1}{#2}\TOres\TOspace in}
% \newcommand\TOcm[2][2]{\setlength{\TOunit}{1cm}\TOop{#1}{#2}\TOres\TOspace cm}
% \newcommand\TOmm[2][2]{\setlength{\TOunit}{1mm}\TOop{#1}{#2}\TOres\TOspace mm}
% \newcommand\TOem[2][2]{\setlength{\TOunit}{1em}\TOop{#1}{#2}\TOres\TOspace em}
% Examples:
% The width of this document is \TOpt[0]{\textwidth}, that is, \TOin{\textwidth}, whereas the height is \TOpt[3]{\textheight}, i.e. \TOin{\textheight}. Here we have some equivalences between different units:
% \begin{center}
%    \renewcommand{\arraystretch}{1.5}
%    \begin{tabular}{r*{4}{@{\ =\ }r}}
%       1pt & \TOpt[0]{1pt} & \TOin[3]{1pt} & \TOcm{1pt} & \TOmm{1pt}  \\
%       1in & \TOpt{1in} & \TOin[0]{1in} & \TOcm{1in} & \TOmm[1]{1in}  \\
%       1cm & \TOpt{1cm} & \TOin[3]{1cm} & \TOcm[0]{1cm} & \TOmm[0]{1cm}  \\
%       1mm & \TOpt{1mm} & \TOin[3]{1mm} & \TOcm{1mm} & \TOmm[0]{1mm}
%    \end{tabular}
% \end{center}
% The em unit depends on which font is active:
% \begin{itemize}
%    \item 1 cm = \TOem{1cm}, 1 em = \TOcm{1em}
%    \item {\bfseries 1 cm = \TOem{1cm}, 1 em = \TOcm{1em}}
%    \item {\bfseries \large 1 cm = \TOem{1cm}, 1 em = \TOcm{1em}}
%    \item {\ttfamily 1 cm = \TOem{1cm}, 1 em = \TOcm{1em}}
% \end{itemize}
%
%
% Copy-Pasteable Listings
% http://www.monperrus.net/martin/copy-pastable-listings-in-pdf-from-latex
% http://tex.stackexchange.com/questions/4911/phantom-spaces-in-listings-pdf
\usepackage[space=true]{accsupp}
\newcommand{\copyablespace}{\BeginAccSupp{method=hex,unicode,ActualText=00A0}\ \EndAccSupp{}}
%
%
% Dummy text
\usepackage{blindtext}
%
% -----------------------------------------------------------------------------
% hyperref
%
\usepackage{hyperref}
%
\hypersetup{
  colorlinks=true,
  linktocpage=true,
  % pdfstartpage=3,
  pdfstartview=FitV,
  breaklinks=true,
  pageanchor=true,
  pdfpagemode=UseOutlines,
  plainpages=false,
  bookmarksnumbered,
  bookmarksopen=true,
  bookmarksopenlevel=1,
  hypertexnames=true,
  pdfhighlight=/O,
  urlcolor=webbrown,
  linkcolor=RoyalBlue,
  citecolor=RoyalBlue,
  % pagecolor=RoyalBlue,
  pdftitle={Meine Diss},                                                        % ******** An eigenen Titel anpassen! ********
  pdfauthor={\textcopyright\ Vorname Nachname},                                 % ******** An eigenen Namen anpassen! ********
  pdfsubject={\finalVersionString},
  pdfkeywords={},
}
\ifthenelse{\boolean{\printversion}}{%
  \hypersetup{
    hidelinks=true,
  }
}{
}
%
% Deutsche Bezeichner für \autoref{}
\addto\extrasngerman{%
  \def\subsectionautorefname{Abschnitt}%
  \def\definitionautorefname{Definition}%
  \def\algorithmautorefname{Algorithmus}%
  \def\subfigureautorefname{Abbildung}%
  \def\tableautorefname{Tabelle}%
}
%
%%%%%%%%%%%%%%%%%%%%%%%%%%%%%%%%%%%%%%%%%%%%%%%%%%%%%%%%%%%%%%%%%%%%%%%%%%%%%%%
% Bugfixes
%
% Fixes error: "Leaders not followed by proper glue." after upgrading from
% Ubuntu 13.10 to 14.04 (see http://tex.stackexchange.com/a/135356)
\makeatletter
\def\hrulefill{\leavevmode\leaders \hrule height \rulethickness \hfill\kern\z@}
\makeatletter
%
%%%%%%%%%%%%%%%%%%%%%%%%%%%%%%%%%%%%%%%%%%%%%%%%%%%%%%%%%%%%%%%%%%%%%%%%%%%%%%%
% Weitere includes
%
%%%%%%%%%%%%%%%%%%%%%%%%%%%%%%%%%%%%%%%%%%%%%%%%%%%%%%%%%%%%%%%%%%%%%%%%%%%%%%%
% Autor: Christian Hinrichs, Januar 2014
%%%%%%%%%%%%%%%%%%%%%%%%%%%%%%%%%%%%%%%%%%%%%%%%%%%%%%%%%%%%%%%%%%%%%%%%%%%%%%%
% Abbreviations and general terms
%
\newcommand{\mydh}{d.\,h.}
\newcommand{\myua}{unter anderem}
\newcommand{\vgl}{vgl.}
\newcommand{\sog}{sog.}
\newcommand{\bspw}{bspw.}
\newcommand{\Bspw}{Bspw.}
\newcommand{\bzgl}{bzgl.}
\newcommand{\Bzgl}{Bzgl.}
\newcommand{\ggf}{ggf.}
\newcommand{\bzw}{bzw.}
\newcommand{\usw}{usw.}
\newcommand{\etc}{etc.}
\newcommand{\zB}{z.\,B.}
\newcommand{\na}{n.\,a.}
\newcommand{\odba}{o.\,B.\,d.\,A.}
\newcommand{\Odba}{O.\,B.\,d.\,A.}
\newcommand{\gdw}{g.\,d.\,w.}
\newcommand{\oae}{o.\,ä.}
%
%%%%%%%%%%%%%%%%%%%%%%%%%%%%%%%%%%%%%%%%%%%%%%%%%%%%%%%%%%%%%%%%%%%%%%%%%%%%%%%
% Acronyms
%
\makeatletter
\newcommand{\mkacr}[2]{%
  % #1 = the acronym
  % #2 = definition of the acronym
  %
  % Example:
  % \mkacr{KWK}{Kraft-Wärme-Kopplung}
  % This will do the following:
  % - Create a command "\KWK" that expands to "\textsf{KWK}"
  % - Create a tooltip over the expanded text,
  %   containing "Definition: Kraft-Wärme-Kopplung"
  % - Create command \xKWK just like above, but without tooltip
  % - Create an entry in the nomenclature (list of acronyms, nomencl package
  %   required) using the string "Kraft-Wärme-Kopplung" from argument #2
  %
  % Define the actual command
  \global\expandafter\DeclareRobustCommand\csname #1\endcsname{%
    \href{Definition: #2}{%
      \textcolor{black}{\textsf{#1}}%
    }%
  }%
  % The same for the "\x..." variant without tooltip
  \global\expandafter\DeclareRobustCommand\csname x#1\endcsname{%
    \textsf{#1}
  }%
  %
  % Finally, create the nomenclature entry (nomencl package required)
  \nomenclature[A#1]{\textsf{#1}}{#2}
}
\makeatother
%
% -----------------------------------------------------------------------------
% Es folgen einige Beispielakronyme zur Veranschaulichung:
%
% Standard Acronyms:
\mkacr{IEEE}{Institute of Electrical and Electronics Engineers}
\mkacr{SCADA}{Supervisory Control and Data Acquisition}
\mkacr{VK}{Virtuelles Kraftwerk}
\mkacr{OFFIS}{Oldenburger Forschungs- und Entwicklungsinstitut für Informatik}
\mkacr{COHDA}{Combinatorial Optimization Heuristic for Distributed Agents}
%
% Sepcials (which require either a special command name, or a special tooltip):
% \newcommand{\PtoP}{%
%   \href{Definition: Peer-to-Peer Netz}{%
%     \textcolor{black}{\textsf{P2P}}%
%   }%
% }
% \nomenclature[AP2P]{\textsf{P2P}}{Peer-to-Peer Netz}
%
%
%
%%%%%%%%%%%%%%%%%%%%%%%%%%%%%%%%%%%%%%%%%%%%%%%%%%%%%%%%%%%%%%%%%%%%%%%%%%%%%%%
% Symbols
%
% For symbols, we usually cannot use the symbol itself as command name, so that
% must be customizable. Also, we want to be able to assign an index to the
% symbols. Thus the command definition below is a lot more difficult.
%
\makeatletter
\newcommand{\mksym}[4]{%
  % #1 = command name for the symbol
  % #2 = the symbol itself
  % #3 = definition of the symbol
  % #4 = definition of the prefix, for nomenclature sorting. May be left empty.
  %
  % Example:
  % \mksym{sfit}{v}{Güte}{11g}
  % This will do the following:
  % - Create a command "\sfit" that expands to "v" in math mode,
  %   or to "$v$" in text mode
  % - Create a command "\sfit[i]" that expands to "v_{i}" or "$v_{i}$",
  %   where "i" is an arbitrary argument
  % - Create a tooltip over the expanded symbol, containing "Definition: Güte"
  % - Create commands \xsfit and \xsfit[i] just like above, but without tooltip
  % - Create an entry in the nomenclature (list of symbols, nomencl package
  %   required) using the string "Güte" from argument #3 and the sorting prefix
  %   from argument #4
  %
  % First, define delegates for optional argument handling
  \global\expandafter\DeclareRobustCommand\csname #1\endcsname{%
    \@ifnextchar[%
      {\csname @@#1\endcsname}% with some argument [foo]
      {\csname @#1\endcsname}% without argument
  }%
  % The same for the "\x..." variant without tooltip
  \global\expandafter\DeclareRobustCommand\csname x#1\endcsname{%
    \@ifnextchar[%
      {\csname @@x#1\endcsname}% with some argument [foo]
      {\csname @x#1\endcsname}% without argument
  }%
  %
  % Second, build the actual delegate targets (hyperref package required)
  \global\expandafter\def\csname @@#1\endcsname[##1]{%
    \ensuremath{%
      \textrm{%
        \href{Definition: #3}{%
          \textcolor{black}{\ensuremath{#2}}%
        }%
      }%
      _{##1}%
    }%
  }%
  \global\expandafter\def\csname @#1\endcsname{%
    \textrm{%
      \href{Definition: #3}{%
        \textcolor{black}{\ensuremath{#2}}%
      }%
    }%
  }%
  \global\expandafter\def\csname @@x#1\endcsname[##1]{%
    \ensuremath{#2_{##1}}%
  }%
  \global\expandafter\def\csname @x#1\endcsname{%
    \ensuremath{#2}%
  }%
  %
  % Finally, create the nomenclature entry (nomencl package required)
  \nomenclature[S#4]{\ensuremath{#2}}{\hspace*{-1.5cm}#3}
}
\makeatother
%
% Declare \mathsfit as sans-serif + non-bold + italic in math mode,
% using the Kp font
\DeclareMathAlphabet{\mathsfit}{\encodingdefault}{jkpss}{m}{it}
%
% -----------------------------------------------------------------------------
% Es folgen einige Beispielsymbole zur Veranschaulichung:
%
% Symbols: Allgemein
\nomenclature[S00]{\textbf{\textsf{Allgemein}}}{}
\mksym{swl}{\mathrm{P}_{\mathrm{el}}}{Elektrische Wirkleistung}{0pel}
\mksym{stl}{\mathrm{P}_{\mathrm{th}}}{Thermische Leistung}{0pth}
%
% Symbols: Chapter 1 (latin chars)
\nomenclature[S10]{}{}
\nomenclature[S11]{\textbf{\textsf{Einleitung (\autoref{sec:einleitung})}}}{}
\mksym{sverbundtn}{a}{Verbundteilnehmer}{11a0}
\mksym{sverbundsp}{\hat{a}}{Verbundsprecher}{11a1}
\mksym{sverbund}{\mathsfit{A}}{Verbund}{11a2}
\mksym{snachbarschaft}{\mathsfit{N}}{Nachbarschaft}{11n}
%
% Symbols: Chapter 1 (greek chars)
\nomenclature[S12]{}{}
\mksym{sbeta}{\beta}{Regressionskoeffizient}{12beta}
\mksym{sstoerung}{\Delta}{Störung}{12delta}
\mksym{sloesungskandidat}{\gamma}{Lösungskandidat}{12gamma}
\mksym{sarbeitsged}{\kappa}{Arbeitsgedächtnis}{12kappa0}
\mksym{skonfiguration}{\Omega}{Konfiguration / wahrgenommener Systemzustand}{32omega1}
\mksym{swlprodukt}{\zeta}{Wirkleistungsprodukt}{12zeta}
%
% Symbols: Chapter 2
% \nomenclature[S20]{}{}
% \nomenclature[S21]{\textbf{\textsf{Konvergenz (\autoref{sec:convergence})}}}{}
% \mksym{sstair}{\mathcal{A}}{Prädikat einer Konvergenzstufe}{2}
% \mksym{sexec}{e}{Schleifenausführung}{2e0}
% \mksym{sexecs}{E}{Parallele Schleifenausführungen}{2e1}
% \mksym{sconf}{c}{Ausführungsfolge}{2c}
% \mksym{sfitconf}{h}{Gütesumme für Ausführungsfolge}{2h}
% \mksym{ssrall}{\mathsfit{L}}{Lösungsraum}{2l}
% \mksym{svarfunc}{V\mskip-\thinmuskip{}F}{Potenzialfunktion für Terminierung}{2vs}
%
%
%%%%%%%%%%%%%%%%%%%%%%%%%%%%%%%%%%%%%%%%%%%%%%%%%%%%%%%%%%%%%%%%%%%%%%%%%%%%%%%
% Simple macros
%
\newcommand{\engl}[1]{\emph{#1}}
\newcommand{\Engl}[1]{\normalfont\emph{#1}}
\newcommand{\todo}[1]{\marginpar{\textcolor{brown}{\textbf{TODO:}\\#1}}}
\newcommand{\verweis}[1]{$\rightarrow$\,#1}
\newcommand*{\defeq}{\mathrel{\vcenter{\baselineskip0.5ex \lineskiplimit0pt
                     \hbox{\scriptsize.}\hbox{\scriptsize.}}}%
                     =}
\DeclareMathOperator*{\argmin}{arg\,min}
% Big operators:
\newcommand{\BIGOP}[1]{\mathop{\mathchoice%
 {\raise-0.22em\hbox{\huge $#1$}}%
 {\raise-0.05em\hbox{\Large $#1$}}{\hbox{\large $#1$}}{#1}}}
\newcommand{\bigtimes}{\BIGOP{\times}}
%
% Math align column for tables
% http://tex.stackexchange.com/a/78839
\newcolumntype{A}{ >{$} r <{$} @{} >{${}} l <{$} } % A for "align"
%% (1) "r" column in math mode:          >{$} r <{$}
%% (2) no space:                         @{}
%% (3) "l" column in math mode, with
%%     an empty subformula at the start: >{${}} l <{$}
%
%
%%%%%%%%%%%%%%%%%%%%%%%%%%%%%%%%%%%%%%%%%%%%%%%%%%%%%%%%%%%%%%%%%%%%%%%%%%%%%%%
% Environments
%
\newtheorem{theorem}{Theorem}
\newtheorem{lemma}{Lemma}
\newtheorem{korollar}{Korollar}
%
\newtheoremstyle{mydefinition}%
  {}%        space above
  {}%        space below
  {}%{\color{primary}}%           body font
  {}%           indent amount
  {\bfseries}%  head font
  {}%           punctuation after head
  {\newline}%   space after head
  {}%           head spec
%
%
\newtheoremstyle{behavior}%
  {}%        space above
  {}%        space below
  {\slshape}%           body font
  {}%           indent amount
  {\slshape}%  head font
  {}%           punctuation after head
  {\newline}%   space after head
  {}%           head spec
%
%
\declaretheorem[numberwithin=section,name=Definition,style=mydefinition,mdframed={
  usetwoside=false,%
  skipabove=\topsep,%
  skipbelow=\topsep,%
  leftmargin=0.1em,%
  rightmargin=0em,%
  topline=false,rightline=false,bottomline=false,leftline=true,%
  % linecolor=primary,%
  linewidth=1.5pt,%
  innerleftmargin=1em,%
  innerrightmargin=0em,%
  innertopmargin=0.1em,%
  innerbottommargin=0.1em,%
  % frametitlerule=true,%
  % frametitlerulecolor=green,%
  % frametitlebackgroundcolor=\examplecolor,%
  % frametitlerulewidth=2pt,%
}]{definition}
%
\declaretheorem[numberwithin=section,name=Beispiel,style=mydefinition,mdframed={
  usetwoside=false,%
  skipabove=1.5\topsep,%
  skipbelow=1.5\topsep,%
  leftmargin=0em,%
  rightmargin=0em,%
  topline=true,rightline=false,bottomline=true,leftline=false,%
  % linecolor=primary,%
  linewidth=1.25pt,%
  innerleftmargin=0em,%
  innerrightmargin=0em,%
  innertopmargin=\topsep,%
  innerbottommargin=\topsep,%
  frametitlerule=true,%
  % frametitlerulecolor=green,%
  % frametitlebackgroundcolor=\examplecolor,%
  % frametitlerulewidth=2pt,%
}]{example}
%
\declaretheorem[name={Lokale Verhaltensregel, Version},style=behavior,mdframed={
  usetwoside=false,%
  skipabove=1.5\topsep,%
  skipbelow=1.5\topsep,%
  leftmargin=0em,%
  rightmargin=0em,%
  topline=false,rightline=false,bottomline=false,leftline=false,%
  % linecolor=primary,%
  linewidth=1.5pt,%
  innerleftmargin=0em,%
  innerrightmargin=0em,%
  innertopmargin=\topsep,%
  innerbottommargin=\topsep,%
  frametitlerule=true,%
  % frametitlerulecolor=green,%
  % frametitlebackgroundcolor=\examplecolor,%
  % frametitlerulewidth=2pt,%
}]{behavior}
%
%
%%%%%%%%%%%%%%%%%%%%%%%%%%%%%%%%%%%%%%%%%%%%%%%%%%%%%%%%%%%%%%%%%%%%%%%%%%%%%%%
% Listings
%
\lstdefinestyle{python}{language=Python,
  basicstyle=\ttfamily\footnotesize,
  % keywordstyle=\color{javaLila}\bfseries,
  % commentstyle=\color{javaGreen},
  % stringstyle=\color{javaBlue},
  % numbers=left,
  % numberstyle=\tiny,
  % stepnumber=1,
  showstringspaces=false,
  captionpos=t,
  breaklines=false,
  % morecomment=*[s][\color{javaDocBlue}]{/**}{*/},
  % tabsize=2,
  % emph={@author, @deprecated, @param, @return, @see, @since, @throws, @version, @serial, @serialField, @serialData, @link},
  % emphstyle=\color{javaDocTags}\bfseries,
  extendedchars=true,
  frame=,
  upquote=true,
  columns=fullflexible,
  keepspaces=true,
  % Copy-Pasteable:
  % http://www.monperrus.net/martin/copy-pastable-listings-in-pdf-from-latex
  % http://tex.stackexchange.com/questions/4911/phantom-spaces-in-listings-pdf
  literate={*}{{\char42}}1
           {-}{{\char45}}1
           {\ }{{\copyablespace}}1
}
%%%%%%%%%%%%%%%%%%%%%%%%%%%%%%%%%%%%%%%%%%%%%%%%%%%%%%%%%%%%%%%%%%%%%%%%%%%%%%%
% TikZ
%
\pgfdeclarelayer{bg}    % declare background layer
\pgfsetlayers{bg,main}  % set the order of the layers (main is the standard layer)
\setvalue{nradius=2.5}
\setvalue{nsize=1cm}
%
\tikzstyle{mynode}=[
  rectangle,
  draw=primary,
  fill=primary,
  rounded corners,
  % drop shadow,
  % text centered,
  % anchor=north,
  text=white,
  % text width=3cm,
  minimum size=\getvalue{nsize},
]
\tikzstyle{mybox}=[
  rectangle,
  draw=primary,
  fill=primary,
  text=white,
  inner sep=1em,
  text width=3cm,
  align=center,
]
\tikzstyle{myline}=[
  -,
  draw=primary,
  line width=2pt,
  % line cap=round,
  % line join=round,
]
\tikzstyle{myconnection}=[
  myline,
  % fill=primary,
  ->,
  % >=triangle 45,
  line width=1.25pt,
]
\tikzstyle{myarrow}=[
  single arrow,
  % single arrow head extend=0.5cm,
  draw=primaryB,
  fill=primaryB,
  minimum height=1cm,
  % minimum width=3cm,
]
\tikzstyle{mycloud}=[
  % cloud ignores aspect,
  cloud callout,
  draw=primary,
  aspect=2,
  callout pointer start size=6pt,
  callout pointer end size=3pt,
]
\tikzstyle{mymsg}=[
  draw=primary,
  line width=1.5pt,
  densely dashed,
  postaction={decorate},
  shorten >=3pt,
]
%
%                                                                   % Lade eigene Stil-Kommandos
%%%%%%%%%%%%%%%%%%%%%%%%%%%%%%%%%%%%%%%%%%%%%%%%%%%%%%%%%%%%%%%%%%%%%%%%%%%%%%%
% Autor: Christian Hinrichs, 2013--2015
%%%%%%%%%%%%%%%%%%%%%%%%%%%%%%%%%%%%%%%%%%%%%%%%%%%%%%%%%%%%%%%%%%%%%%%%%%%%%%%
\hyphenation{
  fahr-plan-basiert
  fahr-plan-basier-te
  fahr-plan-basier-ten
  Selbst-or-ga-ni-sa-tions-stra-te-gien
  Such-raum-modell
  Pla-nungs-ho-ri-zont
  Sys-tem-ele-men-ten
  Funk-tions-wert
  Ak-tion
  Re-ak-tions-zeit
  Eva-lu-ation
  Ab-schnit-te
  Re-gres-sions-tests
  Sze-na-rien-de-fi-ni-tion
  Wirk-leis-tungs-er-brin-gung
  typ-spe-zi-fi-sche
  mo-dell-spe-zi-fi-scher
  typ-un-ab-hän-gig
}                                                             % Lade Silbentrennungs-Definitionen
%                                                                 % Lade Präambel
  \begin{document}
  \selectlanguage{ngerman}
  \pagestyle{scrheadings}                                                       % Aktiviere Kopfzeilen
  % \setcounter{chapter}{0}                                                       % Korrigiere Kapitelzähler
\fi
%
%%%%%%%%%%%%%%%%%%%%%%%%%%%%%%%%%%%%%%%%%%%%%%%%%%%%%%%%%%%%%%%%%%%%%%%%%%%%%%%
%
%
\chapter{Einleitung}\label{sec:einleitung}
%
Dieses Template entstand im Rahmen der Anfertigung meiner Dissertation \cite{Hinrichs2014} und hat zum Tiel, ein optisch möglichst ansprechendes Dokument zu erzeugen, welches dabei die derzeit geltenden typografischen Regeln soweit wie sinnvoll und möglich beachtet. Zugleich beinhaltet es eine Menge von technischen Finessen, welche einerseits das Arbeiten am Dokument erleichtern, und andererseits dem Leser später gewisse Hilfestellungen geben. Im Folgenden werden die wesentlichen Punkte zur Verwendung des Templates kurz erläutert. Dabei werden zumeist Beispiele gegeben, um das Resultat der entsprechenden Funktionen aufzuzeigen. Das zweite Kapitel beinhaltet weiteren Blindtext zur Visualisierung einiger Gestaltungselemente wie Überschriften und Aufzählungen.
%
\par Generell gilt, dass neben dieser PDF auch die \texttt{tex}-Dateien betrachtet werden sollten, um einen Einblick in die korrekte Verwendung der vorgestellten Kommandos und Funktionen zu erhalten.
%
\begin{itemize}
  \item figure, subfigure, footnote
\end{itemize}
%
%
\section{Dateistruktur des Templates}\label{sec:struktur}
%
Aufgrund des Umfangs wurde das Template stark modularisiert. Die Hauptdatei ist \texttt{main.tex}, in welcher alle weiteren Inhalte zusammenlaufen und zu einem Gesamtdokument verknüpft werden:
%
\begin{description}
  \item[\texttt{setup.tex}] Beinhaltet das wesentliche typografische Layout (Dokumentenklasse, Satzspiegel, Fonts) sowie alle zu ladenden Pakete (microtype, biblatex, hyperref, tikz und viele mehr). Am Ende der Datei werden die Dateien \texttt{style.tex} und \texttt{hyphenation.tex} inkludiert.
  \item[\texttt{style.tex}] Hier werden Textelemente wie Abkürzungen, Akronyme und Symbole sowie weitere Stilelemente wie Definitionen, Beispiele und TikZ-Stile definiert.
  \item[\texttt{hyphenation.tex}] Stellt manuell eingetragene Trennregeln für fachspezifische Begriffe zur Verfügung, die ansonsten falsch oder an ungewollten Stellen getrennt werden. Für In-Text-Befehle zum Trennen von zusammengesetzten Wörten wie etwa TikZ"=Abbildungen siehe die Dokumentation von \emph{ngerman} (Abschnitt 2.2.4 in Version 2.5 von \texttt{gerdoc.pdf}).
  \item[\texttt{title.pdf}] Titelblat, wird aufgrund des abweichenden Satzspiegels in einem separaten Dokument von \texttt{title.tex} erzeugt.
  \item[\texttt{abstract-de.tex}, \texttt{abstract-de.tex}] Zusammenfassung der Arbeit in deutscher und englischer Sprache.
  \item[\texttt{01-einleitung.tex}, \texttt{02-blindtext.tex}] Eigentlicher Inhalt der Arbeit, nach Kapiteln aufgeteilt.
  \item[\texttt{appendix-a.tex}] Erster Anhang.
  \item[\texttt{literatur.bib}] Literatur-Datenbank, zitierte Elemente werden zu einem Literaturverzeichnis am Ende der Arbeit kompiliert.
\end{description}
%
Desweiteren steht mit \texttt{tikz.tex} ein Minimaldokument zum Testen von TikZ"=Abbildungen bereit. Neben der Hauptdatei können wahlweise alle Inhalts-Dateien (Titelblatt, die einzelnen Kapitel, und auch \texttt{tikz.tex}) auch separat kompiliert werden. Dies ist insbesondere bei umfangreichen Arbeiten sinnvoll, um während der Anfertigung die wiederholt erforderlichen Kompiliervorgänge zu beschleunigen.
%
%
\section{Kompilieren}\label{sec:kompilieren}
%
Zum Kompilieren unter Linux steht ein Bash-Skript zu Verfügung (\texttt{compile.sh}), welches alle erforderlichen Kompilier-Schritte in intelligenter Weise ausführen kann. Die im Verzeichnis \texttt{tikz/} vorhandenen TikZ"=Abbildungen werden dabei parallel auf allen verfügbaren CPU-Kernen kompiliert und anschließend automatisch als PDF in das Gesamtdokument inkludiert (siehe PGF"=Dokumentation, Kapitel~32, \enquote{Externalization Library}). Die Aufrufsyntax des Skriptes ist wie folgt:
%
\begin{center}
  \texttt{./compile.sh [tex-file [full | lazy]]}
\end{center}
%
Wird das Skript ohne Argument aufgerufen, so wird das Gesamtdokument von Grund auf gebaut. Es kann jedoch auch eine einzelne Zieldatei angegeben werden (\zB{} \texttt{01-einleitung.tex}), dann wird diese separat kompiliert. Als optionales zweites Argument kann \texttt{full} oder \texttt{lazy} angegeben werden: \texttt{full} bewirkt, \ggf{} vorhandene temporäre Dateien aus vorhergehenden Läufen zunächst gelöscht werden, bevor die Zieldatei dann von Grund auf vollständig erstellt wird. Das Argument \texttt{lazy} bewirkt das Gegenteil, hier wird nur ein einzelner Lauf durchgeführt, wobei vorhandene temporäre Dateien soweit wie möglich wiederverwendet werden. Beispiele:\\[1em]
%
\texttt{./compile.sh}\\Erstelle das Gesamtdokument, beginnend mit \texttt{main.tex}. Ist eine aktuelle \texttt{.bbl} Datei vorhanden, so wird diese für das Literaturverzeichnis wiederverwendet, und der \texttt{biblatex} Vorgang wird übersprungen. Gleiches gilt für TikZ"=Abbildungen im Verzeichnis \texttt{tikz/}: Existiert ein aktuelles Kompilat der TikZ"=Abbildungen in \texttt{images/}, so werden diese wiederverwendet. Am Ende wird aufgeräumt, indem alle temporären Dateien gelöscht werden.\\[1em]
%
\texttt{./compile.sh tex-file}\\Wie oben, jedoch mit \texttt{tex-file} als Einstiegspunkt statt \texttt{main.tex}.\\[1em]
%
\texttt{./compile.sh tex-file full}\\Wie oben, jedoch werden alle temporären Dateien (Inhaltsverzeichnis, Nomenklatur \etc{}) sowie alle Zwischenkompilate ignoriert und neu erzeugt. Gilt insbesondere auch für das Literaturverzeichnis und alle TikZ"=Abbildungen.\\[1em]
%
\texttt{./compile.sh tex-file lazy}\\Im Gegensatz zu oben werden alle möglichen vorhandenen Dateien wiederverwendet, sofern sie aktuell sind. Temporäre Dateien werden nach dem Kompiliervorgang \textbf{nicht} gelöscht.\\[1em]
%
\textbf{Achtung:} Das Skript geht aktuell davon aus, dass das Projekt in einem Mercurial-Repository liegt! Das Skript prüft mit entsprechenden \texttt{hg} Kommandos, ob Dateien gegenüber dem Stand im Repository modifiziert wurden und neu kompiliert werden müssen.
%
%
\section{Akronyme, Symbole}\label{sec:symbole}
%
Die im Text verwendeten Akronyme (\zB{} \SCADA{}) und Symbole (\zB{} \sstoerung{}) sind im Hinblick auf konsistente Verwendung zentral in \texttt{style.tex} definiert. Unter Verwendung relativ komplexer Kommandos wird hier ermöglicht, mit dieser zentralen Definition zugleich automatisch Einträge in das entsprechende Akronym- und Symbolverzeichnis anzulegen. Zudem können die Akronyme und Symbole mit frei definierbaren Tooltips hinterlegt werden (bei den oben aufgeführten bereits geschehen -- Mouseover zum Testen!).
%
%
\section{Abkürzungen, Makros}\label{sec:makros}
%
Auch Abkürzungen werden aus Konsistenzgründen zentral in \texttt{style.tex} verwaltet: \zB{}, \sog{}, \bzgl{} und weitere. Zudem existieren einige hilfreiche Makros, \bspw{} um englische Begriffe hervorzuheben: \engl{constraint}, oder um eine TODO-Notiz am Seitenrand zu hinerlegen\todo{wie etwa diese hier}.
%
%
\section{Textauszeichnungen}\label{sec:textauszeichnungen}
%
Für Anführungszeichen wird das Paket \emph{enquote} verwendet, welches je nach Spracheinstellung korrekte Anführungszeichen produziert, wie etwa \enquote{diese} im deutschen Kontext. Eingerückte Zitate können mit der \emph{quote} Umgebung erzeugt werden:
%
\begin{quote}
  Zur Platzierung der Quelle und weiteren Optionen siehe die Dokumentation des hier verwendeten Paketes \emph{csquotes}.
\end{quote}
%
Beispiele für Listings und Algorithmen sind in \autoref{app:example} zu finden.
%
%
\section{Überschriften}\label{sec:ueberschriften}
%
Kapitel- und Abschnittsüberschriften sowie Bild-, Listings- und Algorithmenbezeichnungen können in einer Lang- und Kurzform existieren. Die Kurzform ist optional und wird mittels optionalem Parameter des entsprechenden \texttt{chapter}, \texttt{section} oder \texttt{caption} Befehls erzeugt, und sollte bei langen Überschriften verwendet werden:
%
\begin{center}
  \verb!\section[Kurze Überschrift]{Lange Überschrift}!
\end{center}
%
Sie wird im Inhalts- \bzw{} Abbildungsverzeichnissen \etc{} verwendet und kommt außerdem im Seitenkopf zum Einsatz. Dies vermeidet unschöne Umbrüche.
%
%
\section{Abbildungen}\label{sec:abbildungen}
%
%
% \begin{figure}
%   \includegraphics[width=\textwidth]{Smart-Nord-Topologieebenen}
%   \caption[Topologieebenen im Forschungsverbund \enquote{Smart Nord}]{Gesamtübersicht der betrachteten Topologieebenen im Forschungsverbund \enquote{Smart Nord}. Autor der Darstellung: \OFFIS{} -- Institut für Informatik\protect\footnotemark, mit Genehmigung verwendet.}
%   \label{fig:Smart-Nord-Topologieebenen}
% \end{figure}
% \footnotetext{\label{fn:offis}\url{http://www.offis.de}, letzter Zugriff \today}
%
%
%
%%%%%%%%%%%%%%%%%%%%%%%%%%%%%%%%%%%%%%%%%%%%%%%%%%%%%%%%%%%%%%%%%%%%%%%%%%%%%%%
% Ende
%
% Der folgende Code wird nur ausgeführt, wenn diese Datei direkt kompiliert
% wird (also ohne main.tex):
\ifx\fulldocument\undefined
  \printnomenclature
  \printbibliography
  \end{document}
\fi
%