%%%%%%%%%%%%%%%%%%%%%%%%%%%%%%%%%%%%%%%%%%%%%%%%%%%%%%%%%%%%%%%%%%%%%%%%%%%%%%%
% Autor: Christian Hinrichs, Januar 2014
%%%%%%%%%%%%%%%%%%%%%%%%%%%%%%%%%%%%%%%%%%%%%%%%%%%%%%%%%%%%%%%%%%%%%%%%%%%%%%%
%
% Exemplarischer Disclaimer bzgl. Layout etc.
% Gesetzt in leichtem Grau, da nicht unmittelbarer Bestandteil der Diss.
%
% Alternative Möglichkeit zur abstract-Umgebung: siehe scrguide, S.65.
%
\KOMAoption{abstract}{false}
\begin{abstract}
  \color{halfgray}%
  %
  % Die im nachfolgenden Text platzierte Referenz auf Hinrichs2014 als
  % Layout-Quelle muss nicht verwendet werden, kann aber gerne :)
  % TODO: URN des bib-Eintrages nach Veröffentlichung korrigieren.
  \noindent Dieses Dokument wurde unter Verwendung von {\KOMAScript} und {\LaTeX} gesetzt. Die verwendeten Fonts sind Bitstream Charter (11\,pt, Fließtext und mathematischer Formelsatz) sowie \textsf{Kp-Fonts} (für serifenlose Elemente wie Überschriften und Akronyme). Die visuelle Gestaltung basiert auf \cite{Hinrichs2014} (mit \emph{ClassicThesis} und \emph{ArsClassica}). Das Literaturverzeichnis wurde von \emph{biblatex} in Verbindung mit \emph{Biber} generiert, während für das Akronym- und das Symbolverzeichnis \emph{nomencl} eingesetzt wurde. Für Zeichnungen kam \emph{PGF/TikZ} zum Einsatz, Diagramme wurden mit \emph{matplotlib} unter \emph{Python} erzeugt. Die verwendeten typografischen Richtlinien zum mathematischen Formelsatz entstammen \cite{Forssman2004}.
  %
  \\\par\noindent In der elektronischen Version dieses Dokumentes sind alle Akronyme sowie alle wichtigen Symbole (\mydh{} solche, die an mehreren Textstellen auftreten) zur leichteren Nachvollziehbarkeit mit einem \engl{Tooltip} ausgestattet, welcher die jeweilige Definition des Akronyms/Symbols zeigt.
  %
  \\\par\noindent Vorname Nachname\\Oldenburg, \today
  %
\end{abstract}
%