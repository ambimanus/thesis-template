%%%%%%%%%%%%%%%%%%%%%%%%%%%%%%%%%%%%%%%%%%%%%%%%%%%%%%%%%%%%%%%%%%%%%%%%%%%%%%%
% Autor: Christian Hinrichs, Januar 2014
%%%%%%%%%%%%%%%%%%%%%%%%%%%%%%%%%%%%%%%%%%%%%%%%%%%%%%%%%%%%%%%%%%%%%%%%%%%%%%%
% Abbreviations and general terms
%
\newcommand{\mydh}{d.\,h.}
\newcommand{\myua}{unter anderem}
\newcommand{\vgl}{vgl.}
\newcommand{\sog}{sog.}
\newcommand{\bspw}{bspw.}
\newcommand{\Bspw}{Bspw.}
\newcommand{\bzgl}{bzgl.}
\newcommand{\Bzgl}{Bzgl.}
\newcommand{\ggf}{ggf.}
\newcommand{\bzw}{bzw.}
\newcommand{\usw}{usw.}
\newcommand{\etc}{etc.}
\newcommand{\zB}{z.\,B.}
\newcommand{\na}{n.\,a.}
\newcommand{\odba}{o.\,B.\,d.\,A.}
\newcommand{\Odba}{O.\,B.\,d.\,A.}
\newcommand{\gdw}{g.\,d.\,w.}
\newcommand{\oae}{o.\,ä.}
%
%%%%%%%%%%%%%%%%%%%%%%%%%%%%%%%%%%%%%%%%%%%%%%%%%%%%%%%%%%%%%%%%%%%%%%%%%%%%%%%
% Acronyms
%
\makeatletter
\newcommand{\mkacr}[2]{%
  % #1 = the acronym
  % #2 = definition of the acronym
  %
  % Example:
  % \mkacr{KWK}{Kraft-Wärme-Kopplung}
  % This will do the following:
  % - Create a command "\KWK" that expands to "\textsf{KWK}"
  % - Create a tooltip over the expanded text,
  %   containing "Definition: Kraft-Wärme-Kopplung"
  % - Create command \xKWK just like above, but without tooltip
  % - Create an entry in the nomenclature (list of acronyms, nomencl package
  %   required) using the string "Kraft-Wärme-Kopplung" from argument #2
  %
  % Define the actual command
  \global\expandafter\DeclareRobustCommand\csname #1\endcsname{%
    \href{Definition: #2}{%
      \textcolor{black}{\textsf{#1}}%
    }%
  }%
  % The same for the "\x..." variant without tooltip
  \global\expandafter\DeclareRobustCommand\csname x#1\endcsname{%
    \textsf{#1}
  }%
  %
  % Finally, create the nomenclature entry (nomencl package required)
  \nomenclature[A#1]{\textsf{#1}}{#2}
}
\makeatother
%
% -----------------------------------------------------------------------------
% Es folgen einige Beispielakronyme zur Veranschaulichung:
%
% Standard Acronyms:
\mkacr{IEEE}{Institute of Electrical and Electronics Engineers}
\mkacr{SCADA}{Supervisory Control and Data Acquisition}
\mkacr{VK}{Virtuelles Kraftwerk}
\mkacr{OFFIS}{Oldenburger Forschungs- und Entwicklungsinstitut für Informatik}
\mkacr{COHDA}{Combinatorial Optimization Heuristic for Distributed Agents}
%
% Sepcials (which require either a special command name, or a special tooltip):
% \newcommand{\PtoP}{%
%   \href{Definition: Peer-to-Peer Netz}{%
%     \textcolor{black}{\textsf{P2P}}%
%   }%
% }
% \nomenclature[AP2P]{\textsf{P2P}}{Peer-to-Peer Netz}
%
%
%
%%%%%%%%%%%%%%%%%%%%%%%%%%%%%%%%%%%%%%%%%%%%%%%%%%%%%%%%%%%%%%%%%%%%%%%%%%%%%%%
% Symbols
%
% For symbols, we usually cannot use the symbol itself as command name, so that
% must be customizable. Also, we want to be able to assign an index to the
% symbols. Thus the command definition below is a lot more difficult.
%
\makeatletter
\newcommand{\mksym}[4]{%
  % #1 = command name for the symbol
  % #2 = the symbol itself
  % #3 = definition of the symbol
  % #4 = definition of the prefix, for nomenclature sorting. May be left empty.
  %
  % Example:
  % \mksym{sfit}{v}{Güte}{11g}
  % This will do the following:
  % - Create a command "\sfit" that expands to "v" in math mode,
  %   or to "$v$" in text mode
  % - Create a command "\sfit[i]" that expands to "v_{i}" or "$v_{i}$",
  %   where "i" is an arbitrary argument
  % - Create a tooltip over the expanded symbol, containing "Definition: Güte"
  % - Create commands \xsfit and \xsfit[i] just like above, but without tooltip
  % - Create an entry in the nomenclature (list of symbols, nomencl package
  %   required) using the string "Güte" from argument #3 and the sorting prefix
  %   from argument #4
  %
  % First, define delegates for optional argument handling
  \global\expandafter\DeclareRobustCommand\csname #1\endcsname{%
    \@ifnextchar[%
      {\csname @@#1\endcsname}% with some argument [foo]
      {\csname @#1\endcsname}% without argument
  }%
  % The same for the "\x..." variant without tooltip
  \global\expandafter\DeclareRobustCommand\csname x#1\endcsname{%
    \@ifnextchar[%
      {\csname @@x#1\endcsname}% with some argument [foo]
      {\csname @x#1\endcsname}% without argument
  }%
  %
  % Second, build the actual delegate targets (hyperref package required)
  \global\expandafter\def\csname @@#1\endcsname[##1]{%
    \ensuremath{%
      \textrm{%
        \href{Definition: #3}{%
          \textcolor{black}{\ensuremath{#2}}%
        }%
      }%
      _{##1}%
    }%
  }%
  \global\expandafter\def\csname @#1\endcsname{%
    \textrm{%
      \href{Definition: #3}{%
        \textcolor{black}{\ensuremath{#2}}%
      }%
    }%
  }%
  \global\expandafter\def\csname @@x#1\endcsname[##1]{%
    \ensuremath{#2_{##1}}%
  }%
  \global\expandafter\def\csname @x#1\endcsname{%
    \ensuremath{#2}%
  }%
  %
  % Finally, create the nomenclature entry (nomencl package required)
  \nomenclature[S#4]{\ensuremath{#2}}{\hspace*{-1.5cm}#3}
}
\makeatother
%
% Declare \mathsfit as sans-serif + non-bold + italic in math mode,
% using the Kp font
\DeclareMathAlphabet{\mathsfit}{\encodingdefault}{jkpss}{m}{it}
%
% -----------------------------------------------------------------------------
% Es folgen einige Beispielsymbole zur Veranschaulichung:
%
% Symbols: Allgemein
\nomenclature[S00]{\textbf{\textsf{Allgemein}}}{}
\mksym{swl}{\mathrm{P}_{\mathrm{el}}}{Elektrische Wirkleistung}{0pel}
\mksym{stl}{\mathrm{P}_{\mathrm{th}}}{Thermische Leistung}{0pth}
%
% Symbols: Chapter 1 (latin chars)
\nomenclature[S10]{}{}
\nomenclature[S11]{\textbf{\textsf{Einleitung (\autoref{sec:einleitung})}}}{}
\mksym{sverbundtn}{a}{Verbundteilnehmer}{11a0}
\mksym{sverbundsp}{\hat{a}}{Verbundsprecher}{11a1}
\mksym{sverbund}{\mathsfit{A}}{Verbund}{11a2}
\mksym{snachbarschaft}{\mathsfit{N}}{Nachbarschaft}{11n}
%
% Symbols: Chapter 1 (greek chars)
\nomenclature[S12]{}{}
\mksym{sbeta}{\beta}{Regressionskoeffizient}{12beta}
\mksym{sstoerung}{\Delta}{Störung}{12delta}
\mksym{sloesungskandidat}{\gamma}{Lösungskandidat}{12gamma}
\mksym{sarbeitsged}{\kappa}{Arbeitsgedächtnis}{12kappa0}
\mksym{skonfiguration}{\Omega}{Konfiguration / wahrgenommener Systemzustand}{32omega1}
\mksym{swlprodukt}{\zeta}{Wirkleistungsprodukt}{12zeta}
%
% Symbols: Chapter 2
% \nomenclature[S20]{}{}
% \nomenclature[S21]{\textbf{\textsf{Konvergenz (\autoref{sec:convergence})}}}{}
% \mksym{sstair}{\mathcal{A}}{Prädikat einer Konvergenzstufe}{2}
% \mksym{sexec}{e}{Schleifenausführung}{2e0}
% \mksym{sexecs}{E}{Parallele Schleifenausführungen}{2e1}
% \mksym{sconf}{c}{Ausführungsfolge}{2c}
% \mksym{sfitconf}{h}{Gütesumme für Ausführungsfolge}{2h}
% \mksym{ssrall}{\mathsfit{L}}{Lösungsraum}{2l}
% \mksym{svarfunc}{V\mskip-\thinmuskip{}F}{Potenzialfunktion für Terminierung}{2vs}
%
%
%%%%%%%%%%%%%%%%%%%%%%%%%%%%%%%%%%%%%%%%%%%%%%%%%%%%%%%%%%%%%%%%%%%%%%%%%%%%%%%
% Simple macros
%
\newcommand{\engl}[1]{\emph{#1}}
\newcommand{\Engl}[1]{\normalfont\emph{#1}}
\newcommand{\todo}[1]{\marginpar{\textcolor{brown}{\textbf{TODO:}\\#1}}}
\newcommand{\verweis}[1]{$\rightarrow$\,#1}
\newcommand*{\defeq}{\mathrel{\vcenter{\baselineskip0.5ex \lineskiplimit0pt
                     \hbox{\scriptsize.}\hbox{\scriptsize.}}}%
                     =}
\DeclareMathOperator*{\argmin}{arg\,min}
% Big operators:
\newcommand{\BIGOP}[1]{\mathop{\mathchoice%
 {\raise-0.22em\hbox{\huge $#1$}}%
 {\raise-0.05em\hbox{\Large $#1$}}{\hbox{\large $#1$}}{#1}}}
\newcommand{\bigtimes}{\BIGOP{\times}}
%
% Math align column for tables
% http://tex.stackexchange.com/a/78839
\newcolumntype{A}{ >{$} r <{$} @{} >{${}} l <{$} } % A for "align"
%% (1) "r" column in math mode:          >{$} r <{$}
%% (2) no space:                         @{}
%% (3) "l" column in math mode, with
%%     an empty subformula at the start: >{${}} l <{$}
%
%
%%%%%%%%%%%%%%%%%%%%%%%%%%%%%%%%%%%%%%%%%%%%%%%%%%%%%%%%%%%%%%%%%%%%%%%%%%%%%%%
% Environments
%
\newtheorem{theorem}{Theorem}
\newtheorem{lemma}{Lemma}
\newtheorem{korollar}{Korollar}
%
\newtheoremstyle{mydefinition}%
  {}%        space above
  {}%        space below
  {}%{\color{primary}}%           body font
  {}%           indent amount
  {\bfseries}%  head font
  {}%           punctuation after head
  {\newline}%   space after head
  {}%           head spec
%
%
\newtheoremstyle{behavior}%
  {}%        space above
  {}%        space below
  {\slshape}%           body font
  {}%           indent amount
  {\slshape}%  head font
  {}%           punctuation after head
  {\newline}%   space after head
  {}%           head spec
%
%
\declaretheorem[numberwithin=section,name=Definition,style=mydefinition,mdframed={
  usetwoside=false,%
  skipabove=\topsep,%
  skipbelow=\topsep,%
  leftmargin=0.1em,%
  rightmargin=0em,%
  topline=false,rightline=false,bottomline=false,leftline=true,%
  % linecolor=primary,%
  linewidth=1.5pt,%
  innerleftmargin=1em,%
  innerrightmargin=0em,%
  innertopmargin=0.1em,%
  innerbottommargin=0.1em,%
  % frametitlerule=true,%
  % frametitlerulecolor=green,%
  % frametitlebackgroundcolor=\examplecolor,%
  % frametitlerulewidth=2pt,%
}]{definition}
%
\declaretheorem[numberwithin=section,name=Beispiel,style=mydefinition,mdframed={
  usetwoside=false,%
  skipabove=1.5\topsep,%
  skipbelow=1.5\topsep,%
  leftmargin=0em,%
  rightmargin=0em,%
  topline=true,rightline=false,bottomline=true,leftline=false,%
  % linecolor=primary,%
  linewidth=1.25pt,%
  innerleftmargin=0em,%
  innerrightmargin=0em,%
  innertopmargin=\topsep,%
  innerbottommargin=\topsep,%
  frametitlerule=true,%
  % frametitlerulecolor=green,%
  % frametitlebackgroundcolor=\examplecolor,%
  % frametitlerulewidth=2pt,%
}]{example}
%
\declaretheorem[name={Lokale Verhaltensregel, Version},style=behavior,mdframed={
  usetwoside=false,%
  skipabove=1.5\topsep,%
  skipbelow=1.5\topsep,%
  leftmargin=0em,%
  rightmargin=0em,%
  topline=false,rightline=false,bottomline=false,leftline=false,%
  % linecolor=primary,%
  linewidth=1.5pt,%
  innerleftmargin=0em,%
  innerrightmargin=0em,%
  innertopmargin=\topsep,%
  innerbottommargin=\topsep,%
  frametitlerule=true,%
  % frametitlerulecolor=green,%
  % frametitlebackgroundcolor=\examplecolor,%
  % frametitlerulewidth=2pt,%
}]{behavior}
%
%
%%%%%%%%%%%%%%%%%%%%%%%%%%%%%%%%%%%%%%%%%%%%%%%%%%%%%%%%%%%%%%%%%%%%%%%%%%%%%%%
% Listings
%
\lstdefinestyle{python}{language=Python,
  basicstyle=\ttfamily\footnotesize,
  % keywordstyle=\color{javaLila}\bfseries,
  % commentstyle=\color{javaGreen},
  % stringstyle=\color{javaBlue},
  % numbers=left,
  % numberstyle=\tiny,
  % stepnumber=1,
  showstringspaces=false,
  captionpos=t,
  breaklines=false,
  % morecomment=*[s][\color{javaDocBlue}]{/**}{*/},
  % tabsize=2,
  % emph={@author, @deprecated, @param, @return, @see, @since, @throws, @version, @serial, @serialField, @serialData, @link},
  % emphstyle=\color{javaDocTags}\bfseries,
  extendedchars=true,
  frame=,
  upquote=true,
  columns=fullflexible,
  keepspaces=true,
  % Copy-Pasteable:
  % http://www.monperrus.net/martin/copy-pastable-listings-in-pdf-from-latex
  % http://tex.stackexchange.com/questions/4911/phantom-spaces-in-listings-pdf
  literate={*}{{\char42}}1
           {-}{{\char45}}1
           {\ }{{\copyablespace}}1
}
%%%%%%%%%%%%%%%%%%%%%%%%%%%%%%%%%%%%%%%%%%%%%%%%%%%%%%%%%%%%%%%%%%%%%%%%%%%%%%%
% TikZ
%
\pgfdeclarelayer{bg}    % declare background layer
\pgfsetlayers{bg,main}  % set the order of the layers (main is the standard layer)
\setvalue{nradius=2.5}
\setvalue{nsize=1cm}
%
\tikzstyle{mynode}=[
  rectangle,
  draw=primary,
  fill=primary,
  rounded corners,
  % drop shadow,
  % text centered,
  % anchor=north,
  text=white,
  % text width=3cm,
  minimum size=\getvalue{nsize},
]
\tikzstyle{mybox}=[
  rectangle,
  draw=primary,
  fill=primary,
  text=white,
  inner sep=1em,
  text width=3cm,
  align=center,
]
\tikzstyle{myline}=[
  -,
  draw=primary,
  line width=2pt,
  % line cap=round,
  % line join=round,
]
\tikzstyle{myconnection}=[
  myline,
  % fill=primary,
  ->,
  % >=triangle 45,
  line width=1.25pt,
]
\tikzstyle{myarrow}=[
  single arrow,
  % single arrow head extend=0.5cm,
  draw=primaryB,
  fill=primaryB,
  minimum height=1cm,
  % minimum width=3cm,
]
\tikzstyle{mycloud}=[
  % cloud ignores aspect,
  cloud callout,
  draw=primary,
  aspect=2,
  callout pointer start size=6pt,
  callout pointer end size=3pt,
]
\tikzstyle{mymsg}=[
  draw=primary,
  line width=1.5pt,
  densely dashed,
  postaction={decorate},
  shorten >=3pt,
]
%
%